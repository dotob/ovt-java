\documentclass[12pt,ngerman]{scrartcl}

%---- Eingabezeichensatz -----------------------------------------------------
\usepackage[latin1]{inputenc}   % Eingabe deutscher Umlaute
                                % Unix/Linux: latin1
                                % Windows: cp1250

%---- Sonstiges --------------------------------------------------------------
\usepackage{graphicx} % Vorbereitung der Graphiken
\usepackage{varioref}
\usepackage{makeidx}
\usepackage{scrpage2}
\usepackage{newclude}
\usepackage{multirow}
\usepackage{ngerman}
\usepackage{listings}
\usepackage{trfsigns}         % for laplace and fourier signs... 
\usepackage{natbib}           % use me for author year style for [] use [square] as option
%\usepackage[numbers,square]{natbib}  % use me for numberstyle citation
\usepackage{longtable}
\usepackage{hyperref}
\hypersetup{
  pdftitle={title},
  pdfauthor={dotob - Sebastian Kr�mer},
  pdfsubject={OVT Software Dokumentation},
  colorlinks=true, 
  linkcolor=blue
}
\areaset[1.5cm]%               % Zusaetzlicher Rand fuer die Bindung
        {16cm}{22cm}         % Textbreite und Hoehe

%---- einige Seiteneinstellungen ---------------------------------------------
\setlength{\parindent}{0em}
\setlength{\itemindent}{0em}
%\setlength{\oddsidemargin}{0cm}
%\setlength{\evensidemargin}{-0.7cm}
\addtolength{\parskip}{0.4em}
\pagestyle{scrheadings}
\setheadsepline{.4pt}
\setfootsepline{.4pt}

%---- Sonstiges --------------------------------------------------------------
\makeindex
\newcommand{\bb}[1]{{\bf #1}}

\lstset{numbers=left, numberstyle=\tiny, stepnumber=4, breaklines=true, basicstyle=\scriptsize, aboveskip=\bigskipamount,
        numbersep=5pt, frame=single, captionpos=b, commentstyle=\color{blue}, showspaces=false}

%---- Bilder --------------------------------------------------------------
\newcommand{\pictpath}{pix}
\newcommand{\mypicture}[3]{{
   \begin{figure}[htb]
   \hspace*{\fill}
   \includegraphics[scale=#3]{\pictpath/#1}
   \hspace{\fill}
   \caption{\label{#1} #2}
   \end{figure}}}

\vrefwarning

\begin{document}

%---- Titelseite ------------------------------------------------------------
%\titlehead{
%	\begin{longtable}[t]{lr}
%		\multirow{3}{6cm}{\includegraphics[scale=0.5]{pix/fh_logo.jpg}}& \sffamily Abteilung J�lich\\
%		                                                          & \sffamily Labor f�r Medizinische Informatik \\
%		                                                          & \sffamily Prof. Dr. rer. nat W. Hillen\\[1em] 
%	\end{longtable}
%	\hrule
%}
\title{OVT Over't�r\\ Software Dokumentation komplett\\Datenbank \& Adminkonsole \& Mafo-Client}
\author{ 
	dotob gbr\\
	\textbf{Sebastian Kr�mer}\\
}
\date{Aachen/Weilburg, \today\\[4em]}
\maketitle

\tableofcontents

%===========================================================================
\section{Einleitung}
In dieser Dokumentation sollen die Grundz�ge der verwendeten Datenbank beschrieben werden. Ebenso beschrieben werden die beiden Java-Applikationen Adminkonsole und Mafo-Client. Die Adminkonsole ist das Werkzeug f�r die Zentrale in Weilburg. Mit ihr werden Datens�tze f�r die Marktforscher zusammengestellt und Honorarabrechnungen f�r die Marktforscher erzeugt. Die Marktforscher haben ihr eigenes Werkzeug, den Mafo-Client. Mit ihm k�nnen sie die vorbereiteten Datens�tze bearbeiten, also Kontakte aus der Datenbank laden, abtelefonieren und die Ergebnisse wieder in die Datenbank zur�ckspeichern.


% datenbank

\section{Die Datenbank - Aufbau und Konzept}
\label{datenbank}
Die Daten der Over't�r werden in einer relationalen Datenbank abgelegt. Momentan wird daf�r der MySQL-Server benutzt. Eine Datenbank besteht aus Tabellen (im folgenden werden Tabellennamen so dargestellt: \texttt{tabellenname}). Eine Tabelle hat Zeilen und Spalten. Die Zeilen hei�en Datens�tze und die Spalten hei�en Felder (im folgenden werden Feldnamen so dargestellt: \textsl{feldname}). Das hei�t die Hierarchie sieht so aus: Datenbank $\Rightarrow$ Tabelle $\Rightarrow$ Datensatz $\Rightarrow$ Feld.

\mypicture{db-schema}{OVT - Datenbank Schema}{0.4}

Das Schema der Datenbank in der die Over't�r Kundendaten abgelegt werden sollen ist in Abbildung \ref{db-schema} gezeigt. Es gibt 14 Tabellen. Die Haupttabelle hei�t \texttt{kunden} und beherbergt die Daten �ber die Kunden, wie zum Beispiel Adressdaten und Daten zu den H�usern, die von den Martdatenermittlern notiert wurden. 

Bestimmte Felder der Datenbank werden durch weitere Tabellen definiert. So ist zum Beispiel die Farbe der Haust�r nicht im Klartext in der Datenbank abgelegt, sondern als Code, der �ber die Tabelle \texttt{farbe} aufgel�st wird. In der \texttt{kunden}-Tabelle steht dann beispielsweise eine \textit{1} im Feld \textsl{haustuerfarbe} und in der Tabelle \texttt{farben} wird diese \textit{1} dann zu \textit{schwarz} umgesetzt. Ebenso wird mit den Feldern \textsl{fassadenart} und \textsl{bearbeitungsstatus} verfahren. 

In jeder Tabelle gibt es ein Feld \textsl{id}. Dieses Feld reicht aus, um den Datensatz eindeutig in der Tabelle zu finden. Alle anderen Felder der Tabelle k�nnen das nicht leisten. So kann zum Beispiel der St�dtename Aachen in mehreren Datens�tzen auftauchen. Die \textsl{id} \textit{1} aber nur in einem. Felder wie \textsl{id} werden Schl�ssel genannt. 

Jedem Kundendatensatz k�nnen nun weitere Datens�tze aus anderen Tabellen zugeordnet werden. So k�nnen jedem Kunden Aktionen zugeordnet werden. Das hei�t es gibt in der Tabelle \texttt{aktionen} keinen, einen oder mehrere Datens�tze, die einem Kunden zugeordnet sind. 

Mit der \texttt{aktionen}-Tabelle verh�lt es sich genauso wie mit der Tabelle \texttt{kunden}. Auch hier werden wieder Felder mit anderen Tabellen aufgel�st. Wichtig zu nennen sind hier die Felder \textsl{ergebnis} und \textsl{marktforscher}. Eine Aktion ist also nicht nur einem Kunden zugeordnet, sondern auch einem Marktforscher. 




% adminkonsole

\section{Die Adminkonsole}
\subsection{Begiffe}
Zuert sollen einige Begriffe erl�utert werden. Den Teil eines Computerprogramms den der Benutzer auf dem Bildschirm sieht wird Benutzeroberfl�che oder kurz Oberfl�che genannt. Auf dieser Oberfl�che gibt es verschiedene Elemente. Die bekanntesten sind sicher die Fenster, von denen Windows seinen Namen hat. Die Fenster beherbergen �blicherweise die Oberfl�chen der Programme. Auch die Adminkonsole hat ein eigenes Fenster. Wird die Adminkonsole gestartet, so sieht das erscheinende Fenster aus wie in Abbildung \ref{adminconsole_start}. Die Oberfl�che besteht nun aus weiteren Elementen, den Widgets. Widgets sind Bereiche der Oberfl�che, die auf Benutzereingaben reagieren. Folgende Widgets werden benutzt:
\begin{description}
	\item[Button] Klickt man auf einen Button werden Aktionen ausgef�hrt. Er hat eine Beschriftung und zwei Zust�nde. Er kann normal und eingedr�ckt sein. Solange die durch das Anklicken des Buttons gestarteten Aktionen ausgef�hrt werden bleibt er eingedr�ckt und kann dann keine weiteren Klicks empfangen. Auch alle anderen Elemente sind in dieser Zeit inaktiv.
	\item[Eingabefeld] Diese Elemente sind zu Texteingabe gedacht. Hier kann Text eingegeben und dargestellt werden. Eingabefelder reagieren meistens nicht auf das, was eingegeben wurde. Dr�ckt man in manchen Eingabefeldern die Eingabetaste, so wird eine Aktion gestartet. Darauf wird dann hingewiesen.
	\item[Checkbox] Eine Checkbox ist ein kleiner Kasten. Klickt man ihn an, so bekommt er einen Haken. Klickt man nochmal darauf geht der Haken wieder weg. So kann man zwischen diesen beiden Zust�nden umschalten. Die Checkbox ist aktiviert, sobald sie einen Haken darstellt.
	\item[Listen] Eine Liste kann aufgeklappt werden und stellt dann eine Auswahl an Werten zu Verf�gung, die man mit der Maus anklicken und somit ausw�hlen kann. Der aktuelle Wert der Liste wird immer dargestellt.
	\item[Tabelle] Eine Tabelle besteht aus Zeilen und Spalten und kann Werte darstellen und manchmal Eingaben empfangen, so �hnlich wie die Eingabefelder. Eine Tabelle kann auch eine Selektion haben. Also eine aktivierte Zeile. Diese wird blau dargetellt. Auf diese selektierte Zeile k�nnen Aktionen ausgef�hrt werden.
\end{description}

\subsection{Aufgabe}
Die Aufgabe der Adminkonsole besteht darin Kontakte zur Bearbeitung durch die Marktforscher zusammenzustellen. Und nach erfolgter Bearbeitung Honorarabrechnungen f�r die Marktforscher zu erzeugen. Sie wird in der Zentrale in Weilburg benutzt.

Es k�nnen immer bis zu 30 Kontakte zu einer Kontaktgruppe zusammegesucht werden. Die Sucheinstellungen f�r diese Gruppen k�nnen auf der Oberfl�che eingestellt werden. Dies wird im n�chsten Absatz beschrieben.

\subsection{Startoberfl�che}
\mypicture{adminconsole_start}{Adminkonsole - Startzustand}{0.6}
In Abbildung \ref{adminconsole_start} ist der Zustand der Oberfl�che direkt nach dem Start der Adminkonsole dargestellt. Eine Reihe von Elementen ist zu sehen (Die Numerirung der Aufz�hlung pa�t zu denen in der Abbildung). Ihre Funktion soll nun beschrieben werden.
\begin{description}
	\item[1:] Fenstertitel mit Programmname und Version. Bei der Fehlersuche kann es n�tzlich sein, diese Versionsnummer zu beachten.
	\item[2:] Eingabefeld f�r die Postleitzahlensuchmaske. Hier kann eine Postleitzahl eingegeben werden, um das Ergebnis der Suche auf diese einzuschr�nken. Statt einer ganzen Postleitzahl kann auch ein Muster angegeben werden. Ein Muster besteht aus Zeichen die im Ergebnis vorkommen sollen und Platzhaltern, f�r die beliebige Zeichen im Ergebnis stehen k�nnen. Zum Beispiel: Wird in das Eingabefeld "`65*"' eingegeben, so werden alle Kontakte mit Postleitzahlen gefunden, die mit 65 anfangen. Oder man macht die Eingabe "`*65"', dann werden alle Kontakte gefunden bei denen die Postleitzahl mit 65 endet. \textit{Wird die Eingabe des Suchmusters mit der Eingabetaste der Tastatur beendet, so wird der Suchvorgang direkt gestartet.}
	\item[3:] Eingabefeld f�r die St�dtesuchmaske. Sie funktioniert genauso wie die Postleitzahlensuchmaske, nur auf das Feld Stadt angewendet.
	\item[4:] Auswahlliste f�r die Kontaktart. Hier kann gew�hlt werden, ob nur Kontakte gesucht werden, die noch nie angerufen wurden oder auch welche die schon kontaktiert wurden.
	\item[5:] Checkbox f�r Terminbeschr�nkung. Ist diese Checkbox aktiviert, so werden dem Ergebnis nur Kontakte zugef�gt, die schon einmal einen Termin hatten.
	\item[6:] Checkbox f�r Auftragsbeschr�nkung. Ist diese Checkbox aktiviert, so werden dem Ergebnis nur Kontakte zugef�gt, die schon einmal einen Auftrag hatten.
	\item[7:] Suchbutton. Er startet die Suche nach Kontakten, die auf das eingegebene Suchmuster passen. Es werden immer nur maximal 30 Kontakte zu einer Kontaktgruppe zusammengefasst.
	\item[8:] Kontaktgruppentabelle. Hier werden die Ergebnisse der Kontaktgruppensuche dargestellt (mehr siehe Abschnitt \ref{adminkoconsoleguifilled}). Diese Tabelle ist beim Start der Adminkonsole immer leer.
	\item[9:] Aktionsbuttons der Kontaktgruppentabelle. Werden in \ref{adminkoconsoleguifilled} n�her beschrieben.
\end{description}

\subsection{Oberfl�che mit Kontakten}
\label{adminkoconsoleguifilled}
\mypicture{adminconsole_filled}{Adminkonsole - Mit Kontakten}{0.6}
In der Abbildung \ref{adminconsole_filled} wird die Oberfl�che dargestellt, nachdem zwei Kontaktgruppen erstellt wurden.

\begin{description}
	\item[1:] Beispiel f�r eine Suchmaske in der Kontakte bei denen die Postleitzahl mit 65 beginnt gesucht werden. Das "`*"' im Stadteingabafeld bedeuet, dass alle St�dte gefunden werden, dass also keine Einschr�nkung vorgenommen wird.
	\item[2:] Die Kontaktgruppentabelle. Hier werden die gefundenen Gruppen dargestellt, sowie das Suchmuster, mit dem sie gefunden wurden. Es kann sein, dass zu einer Gruppe nicht 30 Kontakte gefunden wurden, deshalb wird die Anzahl der gefundenen Kontakte pro Gruppe in der letzten Spalte dargestellt. Es gibt in der Tabelle immer eine selektierte Zeile (Gruppe). Manche Aktionen, wie l�schen und anzeigen beziehen sich auf die selektierte Zeile (Gruppe).
	\item[3:] Button \texttt{Kontaktgruppe anzeigen}. Eine Liste der ausgew�hlten Kontakte wird angezeigt. Sie kann mit "`Ok"' wieder geschlossen werden.
	\item[4:] Button \texttt{Kontaktgruppe entfernen}. Die selektierte Kontaktgruppe wird aus dem Ergebnis entfernt.
	\item[5:] Button \texttt{Kontaktgruppen f�r Mafo Ablegen}. Wird dieser Button bet�tigt, so wird eine Liste angezeigt mit Marktforschern. Hier soll der Marktforscher ausgew�hlt werden, f�r den die Kontakte bestimmt sind. Dies wird dann in der Datenank gespeichert und der Marktforscher kann diese Kontakte zur Bearbeitung abrufen (siehe auch Abschnitt \ref{verfahren}).
\end{description}

\subsection{Verfahren}
\label{verfahren}
Hier soll kurz dargestellt werden, was im Hintergrund bei der Kontaktgruppenzusammenstellung passiert. 

Zuerst soll das Feld \textsl{bearbeitungsstatus} in der \texttt{kunden}-Tabelle erl�utert werden (siehe Abschnitt \ref{datenbank}). In diesem Feld wird gespeichert, ob ein Kontakt gerade frei ist, tempor�r gesperrt, f�r einen Marktforscher reserviert, bei einem Marktforscher in Bearbeitung oder der Marktforscher die Bearbeitung des Kontakts beendet hat und der Kontakt nun abgerechnet werden kann. Dieses Feld hat gro�e Bedeutung f�r den ganzen Ablauf der Programme Adminkonsole und Mafo-Client.

Die Adminkonsole arbeitet direkt auf der Datenbank. Wird ein Suchmuster eingegeben und der "`30 Kontakte hinzuf�gen"'-Button gedr�ckt, so werden in der Datenbank Kontakte gesucht und eine neue Zeile in die Tabelle hinzugef�gt. Die gefundenen Kontakte werden als tempor�r gesperrt vermerkt, damit sie nicht nochmal gefunden werden. Hat man die Zusammenstellung beendet und den Button "`Kontaktgruppen in der Datenbank f�r MaFo markieren"' gedr�ckt, so werden nach Auswahl des Marktforschers das Feld \textsl{bearbeitungstatus} auf "`zum Marktforscher"' und das Feld \textsl{marktforscher} auf den ausgew�hlten Marktforscher gesetzt und dies bei allen ausgew�hlten Kontakten. So kann der Mafo-Client alle Kontakte suchen, die den \textsl{bearbeitungstatus} auf "`zum Marktforscher"' eingestellt haben und einen bestimmten Marktforscher. Diese sind dann zur Bearbeitung vorgesehen.

Beim Starten und Beenden der Adminkonsole werden alle tempor�ren \textsl{bearbeitungsstatus}-Felder auf frei gesetzt!! Es ist aufgrund dieser Verfahrensart nicht m�glich gleichzeitig mit zwei Adminkonsolen zu arbeiten.

% mafo-client


\end{document}
