% datenbank

\section{Die Datenbank - Aufbau und Konzept}
\label{datenbank}
Die Daten der Over't�r werden in einer relationalen Datenbank abgelegt. Momentan wird daf�r der MySQL-Server benutzt. Eine Datenbank besteht aus Tabellen (im folgenden werden Tabellennamen so dargestellt: \texttt{tabellenname}). Eine Tabelle hat Zeilen und Spalten. Die Zeilen hei�en Datens�tze und die Spalten hei�en Felder (im folgenden werden Feldnamen so dargestellt: \textsl{feldname}). Das hei�t die Hierarchie sieht so aus: Datenbank $\Rightarrow$ Tabelle $\Rightarrow$ Datensatz $\Rightarrow$ Feld.

\mypicture{db-schema}{OVT - Datenbank Schema}{0.4}

Das Schema der Datenbank in der die Over't�r Kundendaten abgelegt werden sollen ist in Abbildung \ref{db-schema} gezeigt. Es gibt 14 Tabellen. Die Haupttabelle hei�t \texttt{kunden} und beherbergt die Daten �ber die Kunden, wie zum Beispiel Adressdaten und Daten zu den H�usern, die von den Martdatenermittlern notiert wurden. 

Bestimmte Felder der Datenbank werden durch weitere Tabellen definiert. So ist zum Beispiel die Farbe der Haust�r nicht im Klartext in der Datenbank abgelegt, sondern als Code, der �ber die Tabelle \texttt{farbe} aufgel�st wird. In der \texttt{kunden}-Tabelle steht dann beispielsweise eine \textit{1} im Feld \textsl{haustuerfarbe} und in der Tabelle \texttt{farben} wird diese \textit{1} dann zu \textit{schwarz} umgesetzt. Ebenso wird mit den Feldern \textsl{fassadenart} und \textsl{bearbeitungsstatus} verfahren. 

In jeder Tabelle gibt es ein Feld \textsl{id}. Dieses Feld reicht aus, um den Datensatz eindeutig in der Tabelle zu finden. Alle anderen Felder der Tabelle k�nnen das nicht leisten. So kann zum Beispiel der St�dtename Aachen in mehreren Datens�tzen auftauchen. Die \textsl{id} \textit{1} aber nur in einem. Felder wie \textsl{id} werden Schl�ssel genannt. 

Jedem Kundendatensatz k�nnen nun weitere Datens�tze aus anderen Tabellen zugeordnet werden. So k�nnen jedem Kunden Aktionen zugeordnet werden. Das hei�t es gibt in der Tabelle \texttt{aktionen} keinen, einen oder mehrere Datens�tze, die einem Kunden zugeordnet sind. 

Mit der \texttt{aktionen}-Tabelle verh�lt es sich genauso wie mit der Tabelle \texttt{kunden}. Auch hier werden wieder Felder mit anderen Tabellen aufgel�st. Wichtig zu nennen sind hier die Felder \textsl{ergebnis} und \textsl{marktforscher}. Eine Aktion ist also nicht nur einem Kunden zugeordnet, sondern auch einem Marktforscher. 


